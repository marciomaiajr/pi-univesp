\chapter{Fundamentação teórica}

\section{Biodigestores}

Os equipamentos utilizados para a produção de biogás são chamados biodigestores. Eles são tanques fechados onde se obtém o gás metano a partir da fermentação de resíduos orgânicos na ausência de oxigênio.

Existem vários tipos de biodigestores, os mais usados são os contínuos e os de batelada (FERRAZ, 1980). No modelo contínuo, é introduzido o material orgânico misturado com água de maneira que o gás metano gerado é armazenado em um gasômetro. Dessa maneira o fluxo de gás é contínuo a medida em que o material orgânico é depositado. Os resíduos são então captados na saída e podem ser utilizados imediatamente. O modelo Indiano e o modelo Chinês são duas implementações diferentes, mas bastante utilizadas do biodigestor do tipo contínuo.

Já no sistema de biodigestor de batelada, o material orgânico é depositado uma só vez e só é removido após todo o gás ter sido utilizado. Este modelo é consideravelmente mais simples que o modelo contínuo.  

Os biodigestores contínuos e de batelada, são conhecidos coletivamente como biodigestores rurais.

\begin{citacao}
“Existem, atualmente uma gama muito grande de modelos de biodigestores, sendo cada um adaptado a uma realidade e uma necessidade de biogás, neste trabalho trataremos exclusivamente de biodigestores utilizados em pequenas propriedades no meio rural”.\cite{abntex2-wiki-como-customizar}
\end{citacao}

Estes são os biodigestores mais comuns e baratos em operação atualmente. O protótipo será construído com base no modelo de batelada.

\section{Biogás}

O biogás é uma mistura gasosa composta por aproximadamente 60\% de gás metano (CH4) e 38\% de dióxido de carbono (CO2). O metano é um gás combustível e pode ser utilizado como fonte energética.

O poder calorífico do biogás é da ordem de 5000 kcal/m3 quando misturado com o dióxido de carbono. Para colocar em perspectiva podemos compará-lo com o poder calorífico de outros combustíveis.

\begin{table}[htb]
\centering
\footnotesize
\begin{tabular}{ll}
\toprule
\multicolumn{2}{c}{\textbf{1$m^3$ de biogás equivale a:}} \\
\midrule \midrule
0,613L & Gasolina \\
\midrule 
0,579L & Querosene \\
\midrule 
0,553L & Óleo Diesel \\
\midrule 
0,454Kg & GLP (Gás liquefeito de petróleo) \\
\midrule 
1,428KW/h & Energia elétrica \\
\bottomrule
\end{tabular}%
\caption{Equivalência de poder calorífico}
\end{table}

1m3 de biogás equivale a:
0,613L	Gasolina
0,579L	Querosene
0,553L	Óleo Diesel
0,454Kg	GLP (Gás liquefeito de petróleo)
1,428KW/h	Energia elétrica
Tabela 1 – Equivalência de Poder Calorífico
O biogás gerado pode ser utilizado diretamente para alimentar fogões ou motores a combustão (como os utilizados em geradores).

\section{Biomassa}

A biomassa é a matéria prima utilizada na produção do biogás. Podem ser divididos em:

\begin{itemize}
\item Esterco (de gado, de suínos, de aves, etc.);
\item Resíduo orgânico residencial;
\item Efluentes (esgoto).
\end{itemize}

É necessário que na fase inicial de operação do biodigestor, seja utilizada uma quantidade de fezes para garantir a presença de bactérias metagênicas, sem as quais o biogás não pode ser produzido (FERRAZ, 1980).

\section{Fatores que afetam o consumo e a produção de gás}

O transporte do biogás não pode ser feito para longas distâncias sem o auxílio de um compressor pois o gás é armazenado em baixa pressão.

A produção do biogás é otimizada em temperaturas ao redor dos 35ºC. Temperaturas abaixo dos 15ºC inibem drasticamente o processo.

\section{Biofertilizante}

Como foi dito, os resíduos da produção de biogás podem ser utilizados como fertilizantes. O nitrogênio é quase que totalmente conservado no processo e o material pode ser utilizado imediatamente.

\section{Aplicação das disciplinas estudadas no projeto integrador}

A disciplina de Produção de Textos foi fundamental para a elaboração deste relatório, fornecendo técnicas importantes de leitura e escrita. Outra disciplina importante foi a de Metodologia Científica, por meio da qual conseguimos entender o processo de construção dos trabalhos científicos. Inglês foi decisivo na hora de consultar referências na língua inglesa. Já a disciplina de Física nos permitiu compreender o comportamento dos gases gerados pelo processo de geração de biogás. Outra matéria fundamental foi a de Introdução à Informática que nos mostrou os principais aplicativos utilizados no meio científico para a confecção de trabalhos acadêmicos. Finalmente cálculo foi responsável por sermos capazes de apreciar as passagens mais complicadas do processo.